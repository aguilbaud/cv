\documentclass[11pt,a4paper]{moderncv}

\usepackage[utf8]{inputenc}

% moderncv themes
\moderncvtheme[blue]{classic}
% d'autres valeurs possibles
%\moderncvstyle{casual}
%\moderncvstyle{oldstyle}
%\moderncvstyle{banking}
% une autre manière de définir la couleur
\moderncvcolor{blue}
% couleurs possibles : 'blue' (par defaut),
% 'orange', 'green', 'red', 'purple', 'grey' et 'black


\usepackage[top=1.1cm, bottom=1.1cm, left=2cm, right=2cm]{geometry}
% Largeur de la colonne pour les dates
\setlength{\hintscolumnwidth}{3.5cm}

% personal data
\name{Adrien}{Guilbaud}
\title{Développement HPC}
\address{3 bis allée Jean Griffon}{31400 Toulouse}{France}
\phone[mobile]{+33 (0)6~15~39~25~18}
\email{adrien.guilbaud33@gmail.com}
\social[linkedin]{adrien-guilbaud}
\extrainfo{23 ans - Permis B}


\begin{document}
\makecvtitle

%Expériences
\section{Expériences}
\cventry{10/2016--Aujourd'hui}{Ingénieur d'études}{Cerfacs}{Toulouse}{}{Amélioration des performances CPU pour les raccords non coïncidents dans le logiciel elsA.\newline{}%
\begin{itemize}%
\item Modification de la gestion des raccords entre les blocs permettant la réduction du nombre de communications MPI;
\item Mise en place d'un ordonnanceur dynamique des messages MPI.
\end{itemize}}

\cventry{04/2016--09/2016}{Stagiaire}{Cerfacs}{Toulouse}{}{Passage en 3D et parallélisation de NTMIX\_CHEMKIN (simulation numérique directe de combustion turbulente).\newline{}%
\begin{itemize}%
\item Modernisation du code/gestion dynamique de la mémoire (Fortran77 vers Fortan90);
\item Développement d'une version 3D;
\item Parallélisation par décomposition et recouvrement de domaine (MPI).
\end{itemize}}

\cventry{2013}{Stagiaire}{Lafon}{Bassens}{}{Développement d’un algorithme de répartition de puissance électrique (SmartGrid)}

%Formation
\section{Formation}
\cventry{2014-2016}{Master}{Université de Bordeaux}{Bordeaux}{\textit{Grade}}{Option CHP - Calcul Haute Performance}
\cventry{2013-2014}{Licence}{Université de Bordeaux}{Bordeaux}{}{}
\cventry{2011-2013}{DUT}{IUT de Bordeaux}{Bordeaux}{}{}

%Compétences
\section{Compétences}
\cvitem{Langages de programmation}{C++, C, Fortran, Java, Python, Bash}
\cvitem{Langages du parallélisme}{MPI, OpenMP, CUDA}
\cvitem{Analyse}{UML, Merise}
\cvitem{Outils}{Emacs, Sublime Text, Eclipse, Git, Gnuplot, \LaTeX}

%Langages
\section{Langages}
\cvitem{Français}{Langue maternelle}
\cvitem{Anglais}{Niveau C1}

%Intérêts
\section{Intérêts}
%\begin{cvcolumns}
%  \cvcolumn{}{\begin{itemize}\item Guitare\item Astronomie\item Parachutisme \end{itemize}}
%\end{cvcolumns}
\cvitem{}{Guitare, Astronomie, Parachutisme}

%Références
\section{Références}
%\begin{cvcolumns}
%  \cvcolumn[0.28]{}{\begin{itemize}\item Marc Montagnac\item Isabelle D'ast \end{itemize}}
%  \cvcolumn[0.32]{}{\begin{itemize}\item +33 (0)5 61 19 30 48\item +33 (0)5 61 19 30 53\end{itemize}}
%  \cvcolumn{}{\begin{itemize}\item marc.montagnac@cerfacs.fr\item isabelle.dast@cerfacs.fr\end{itemize}}
%\end{cvcolumns}

\cventry{}{Marc Montagnac}{Ingénieur Logiciel HPC}{}{}{marc.montagnac@cerfacs.fr - +33~(0)5~61~19~30~48}
\cventry{}{Isabelle D'ast}{Ingénieur de recherche}{}{}{isabelle.dast@cerfacs.fr - +33~(0)5~61~19~30~53}
\end{document}
